\section{Introduction}

This report, prepared as part of the Statistical Modelling and Forecasting master's course,
explores the application of modern statistical techniques within the R programming environment,
emphasizing predictive analytics and forecasting. The course equips graduate students with
advanced statistical tools for analyzing real-world data
and fosters skills to understand professional literature and conduct independent research.

The report is structured into three main sections:
\begin{itemize}
\item \textbf{Model Selection}: Utilizing a Body Mass Index (BMI) data set, this section
  demonstrates the process of selecting the most appropriate statistical model using Generalized
  Additive Models for Location, Scale, and Shape (GAMLSS) techniques covered in the course.
\item \textbf{Centile Estimation}: This part focuses on estimating centiles from the handgrip (HG) strength data set,
  applying the methodologies discussed in the syllabus, particularly emphasizing the flexibility and
  utility of GAMLSS in handling complex distributions.
\item \textbf{Forecasting Application}: The final section will apply the acquired modeling skills to
  a new dataset (to be supplied), showcasing the ability to perform accurate forecasting and predictive
  analysis as a practical demonstration of the theoretical knowledge gained.
\end{itemize}

This structured approach not only reflects the course objectives but also demonstrates practical
competence in the use of advanced statistical techniques for effective data analysis and forecasting.
