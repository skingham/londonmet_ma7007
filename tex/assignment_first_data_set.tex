\section{First data set}

Fitting distributions to the data.  The data set represents BMI data collected on \ldots

BMI is a number calculated from a person's height in meters and weight in kilograms:
\begin{equation}
  BMI = \frac{kg}{m^2}
\end{equation}

\subsection{Comment on the different distributions you are using.}

We are asked to interpret the bmi information of subjects within a one year
age group.  From the summary information we are looking at, the response
variable of `bmi` is a continuous random variable.

As there is no definition of the upper bounds of a person's height or weight,
there's no natural justification to normalise the value of `bmi`, so we are
interested in families of continuous distributions over the range $R_Y = (0, \inf)$.

\begin{figure}[!ht]
\begin{verbatim}
      age              bmi       
 Min.   : 0.030   Min.   :11.17  
 1st Qu.: 1.863   1st Qu.:15.96  
 Median :10.450   Median :17.45  
 Mean   : 9.291   Mean   :18.03  
 3rd Qu.:15.130   3rd Qu.:19.60  
 Max.   :21.700   Max.   :35.42  
\end{verbatim}
\caption {Summary information on the `dbbmi` data}
\end{figure}


\subsection{Which distribution did you choose?}

The \textbf{`Box-Cox Cole Green`} (BCCG) distribution was selected for this analysis.

(c) Give reasons why you chose the distribution in part (b).

The GAMLSS function `fitDist` fits all relevant parametric `gamlss.family` distributions
to a single data vector, choosing the final marginal distributions with the lowest
Akaike Information Criteria (AIC) value with a penality of $k=2$.

(d) Plot the fitted distribution and comment.

(e) State the fitted parameter values of the final chosen model


